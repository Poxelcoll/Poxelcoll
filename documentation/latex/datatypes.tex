
\subsection{General data types}

The library provides a number of general data types.
By "data type" is meant basic data containers that
represent different concepts and support different
operations. For instance, a mathematical vector contains a couple
of coordinates, and supports operations like cross-product
and dot-product. Having a precise understanding of the data types
in the library makes it easier to use the libray.

A general note about the data types in the library is that they
are, in general, immutable. This makes it generally easier
to reason about the library and ensure that it is robust
and correct.

The most important of the data structures in regards to using
the library is the BinaryImage, Mask and CollisionInfo.
If polygons are used as primitives, it will be necessary to use
ConvexCCWPolygon as well.

\textbf{BinaryImage} represents an image. By an 'image' is meant
a 2-dimensional finite area of values, that have a width and a height.
A 'binary image' is an image that holds binary values, meaning that
each value in the image is either 'on' or 'off', or alternatively,
1 or 0. In the library, binary images are used as the primitive for
pixel-perfect collision detection. 'on' or 1 represents that there
is something, while 'off' or 0 represents that there is nothing.
When the binary images of two collision objects overlap,
and there is at least one point in the overlap where both of the
binary images is 'on' or 1, there is a collision. If not, there
is no collision.

The way to construct a pixel-perfect binary image from an image
depends on the image, but usually the transparent/background pixels
are mapped to 'off', while the foreground/not transparent pixels are mapped
to 'on'. Done this way, if the image collides visually with something,
it collides with that same thing in the collision system.

There is a basic abstraction for factories for binary images,
which takes a sequence of rows of 'on'/'off' values, ie. boolean values.

\textbf{Mask} is the main abstraction for collision primitives.
It consists of a origin, an axis-aligned bounding box,
a nonempty convex hull, and an optional binary image
(for more information on what a convex hull is, see wikipedia).

As of version 0.1, it has two potential meanings:
If it has a binary image, it is a pixel-perfect primitive
using the binary image for the pixel-perfect part and
using the axis-aligned bounding box and the convex hull
for bounding volume collision pruning
(for example, if the bounding boxes of two collision primitives does not
overlap, the more expensive collision checking is skipped).
In that case,
both the axis-aligned bounding box and the convex hull
should be precise or over-approximating,
and never under-approximating the binary image.

If instead it does not have a binary image,
it means that it is a polygon-based primitive
using the convex hull for the polygon
and using the axis-aligned bounding box for
bounding volume collision pruning.
The bounding box should be precise or over-approximating
the convex hull, and never under-approximating the
convex hull.

The origin point indicates for the three other parts
(the optional binary image, the convex hull,
and the axis-aligned bounding box).
So, if a point in the mask has position (1, 2),
and the origin point is (5, 5), the effective
position of the point in the mask is (-4, -3).
Setting the origin point the right place is important.
It is the point about which rotations happen, and if
a collision mask is placed at global position (0, 0),
then the point in the mask that correspons to the
origin will be found at global position (0, 0).
For more information regarding coordinates and points,
see P.

Finally, it should be noted that Mask is meant
to be precomputed in some ways. Some parts of it,
such as the convex hull, can take O(n log n) time
to compute,
where n is the number of pixels in the binary image.

\textbf{CollisionInfo} represents a collision object
in a specific state.
It contains a collision primitive,
a position, an angle for rotation, a scaling factor
in the x-axis and the y-axis, and finally the id
of the collision object.

The order of transformation for finding the final positions
of the collision primitives pixels is: origin, scaling, rotation, position.
First the pixels are translated according to the origin,
then they are scaled along the x-axis and the y-axis.
Then they are rotated by the angle, and finally
translated by the position.

For no transformations, the scaling factors should both be
1.0, and the rotation should be 0.0. The translation
vector should be (0.0, 0.0).

The angle is measured in radians, the position in pixels,
and the factors are percentages - so 1.0 is no scaling,
2.0 is double as big along the axis, and 0.5
is half as big along the axis.

\textbf{P} represents a simple 2D-point,
with an x-coordinate and a y-coordinate.
The x-coordinate goes along the 1. or horizontal
axis, and the y-coordinate goes along the 2. or
vertical axis. The units are generally pixels.

In some places of the API, a simple 2D-point
is viewed instead as a 2D-vector. For that reason,
P supports several operations for vectors,
such as cross-product and dot-product.

\textbf{IP} is simply a less feature-full
integer-version of P.

\textbf{BoundingBox} is an axis-aligned bounding box.
It consists of two points, of which the minimum
represents the corner of the axis-aligned box
that is the closest to origo, and the maximum
represents the corner of the axis-aligned box
that is the furthest away from the origo.

BoundingBox supports efficient intersection
checking. It is used as the most inaccurate
bounding volume: it generally over-approximates
a lot, but is very quick, making it useful for
pruning collision objects that are far away from
each other quickly.

\textbf{CollisionPair} is a simple representation
for two collision objects that collide. Since a
collision object cannot collide with itself,
the ids for a CollisionPair may never be the same.
For consistency and comparison, the first id
must always be strictly smaller than the second id.

\textbf{ConvexCCWPolygon} is the superclass for representing
convex polygons. Convex polygons are especially suited for
representing convex hulls. ConvexCCWPolygon has 4 concrete
subclasses, which are disjoint and which together represents
all legal ConvexCCWPolygon's: Empty, Point, List, and Polygon.

\textit{Empty} is the convex polygon that represents no points.
It is mainly included to ensure that all cases are considered
through the type system.

\textit{Point} is the convex polygon that represents exactly one point.

\textit{Line} is the convex polygon that represents a line segment,
described by two points, both of which are included in the line segment.
The points may NOT be the same. For the case where they are the same,
Point must be used.

\textit{Polygon} is the convex polygon that represents a polygon with
non-zero area. Thus, it has at least 3 points, none of which are collinear.
There is a number of important restrictions on Polygon. The first is that
it has 3 or more points. The second is that none of the points are equal.
The third is that the polygon they represent is 'simple'
(see http://en.wikipedia.org/wiki/Simple\_polygon for more information).
The fourth is that the polygon they represent is convex.
The fifth is that the polygon is defined counter-clockwise, such that
if the first three points are named p1, p2 and p3 in that order,
and v1 is the vector |P1 P2|, and v2 is the vector |P2 P3|,
then the cross product of v1 and v2 will be strictly positive.
The sixth is that none of the points are collinear.
The seventh is that the area of the polygon is non-zero.
If these restrictions are not obeyed, the library's efficiency,
accuracy or correctness cannot be guaranteed.
One simple way to construct a correct convex polygon is simply to
make a couple of points where the corners of the polygon should be,
and then use the convex hull algorithm defined in the library
under the name ConvexHull to get the corresponding correct polygon.
The ConvexHull algorithm is robust, and will always give a valid result
no matter the input.

In some cases, a subset of the different polygon cases are needed.
Examples of this includes the \textit{NonemptyConvexCCWPolygon},
\textit{EmptyPoint} and \textit{EmptyPointLine}.
The first includes all concrete types except Empty,
the second includes only Empty and Point, and the third includes
Empty, Point and Line.

