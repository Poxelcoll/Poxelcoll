
\subsection{Pixel-perfect collision detection}

When two collision primitives based on binary images collide,
their bounding boxes are checked for collisions, and if there
is a collision, the intersection of their convex hulls is first found.
If the intersection is non-empty, the pixel-perfect collision detection
begins.

The overall method is to find all the pixels touched or contained
in the intersection, and then check whether for any of these pixels,
the corresponding pixel in each image is 'on'. If there exist at least
one such pixel, then there is a collision, else there is not.

In order to find the corresponding pixels, it is important to note
that in order to go from the coordinate system of an image
to the coordinate system of the intersection,
the image must be transformed according to its transformation matrix.
The transformation matrix handles both origin, scaling, rotation
and translation. So, in order to go the other way, namely going from
the coordinate system of the intersection to the coordinate system
of the image, the inverse transformation must be found.
This is done by finding the inverse matrix of the transformation
matrix, which happens to handle the desired inverse transformation.
So, given a position in the intersection's coordinate system, $p_{intersect}$,
we take the inverse matrix of image 1, ${M_1}^{-1}$, and calculate ${M_1}^{-1} * p_{intersect}$
to find the corresponding position in the image's coordinate system, $p_{image1}$.
Once the corresponding position in the image's coordinate system has been found,
binary image 1 is queried to see if the pixel is 'on' there.
The same process is repeated with image 2.
If both the pixels at positions $p_{image1}$ and $p_{image2}$ in
respectively image 1 and image 2 is 'on', there is a collision,
and the pixel-perfect collision detection is done.

This process is somewhat costly, but because only the intersection
is investigated, not all of both of the images' pixels need to be transformed,
only the intersection's pixels, and then only if a 'hit' (ie. two 'on'-pixels at the same
corresponding position) is not found early,
and finding a 'hit' early is generally more likely for the relatively smaller intersection
than it is for transforming one image and checking each of its pixels.
The chance of finding a 'hit' early depends on how accurately the convex hulls
cover the images. If the convex hulls are accurate, a 'hit' should be found fairly
quickly, and generally only a small part of the intersection needs to be checked.
If the convex hulls are not accurate, most or the whole intersection needs to be checked.
This is the main reason that the convex hulls should be accurate.
If all convex hulls used are very accurate, the efficiency of the collision detection
should be on the level of polygon collision detection, despite having pixel-perfect
precision and ease as well as supporting basic transformations.

As for collision detection between image-based primitives and convex polygon
primitives, the check for the polygon primitive is simply assumed to be true,
since all of the intersection is part of the polygon.
In the case where both primitives are convex polygons, it is only checked
whether the intersection is empty or not; if it is not empty, there is no collision,
since if the intersection is empty the polygons does not overlap and thus cannot
collide.

