
\subsection{History}

This provides an overview of the development process of the library.

\subsubsection{The beginning and version 0.1}

The library was first motivated by the lack of efficient pixel-perfect collision detection
support, especially in the FOSS community. Many solutions are naive,
simply iterating through all the pixels of one image and testing with the other,
and does not support basic transformations. This is not very flexible or
efficient. Other are slightly more sophisticated,
and supports finding the intersection between axis-aligned bounding boxes.
However, axis-aligned bounding boxes are not a tight fit in a lot of common
cases, and they become even less tight when rotated. In some cases, the implementations
are supported only for languages on certain platforms, such as Flash. Examples include
PMASK (C library) and Collision Detection Kit (ActionScript 3.0 library).
Especially this link was somewhat depressing,
\url{http://stackoverflow.com/questions/5138783/c-2d-pixel-perfect-collision-detection-libraries},
since it clearly seems to implicate that no good pixel-perfect collision detection library exists in C++,
even though C++ is popular for games and computational geometry,
and pixel-perfect collision detection is a library/engine-level
part of games.

The original author of this library decided that the next project
to work on might as well be pixel-perfect collision detection,
and once some basic research into the status of the field was done
(to ensure that no good solution actually existed),
the prototyping was begun. With a background in computer graphics,
algorithms, data structures, game development, mathematics,
linear algebra, image analysis and geometry,
the author had a skill set suitable for completing this challenge.

The basic idea was found fairly early, namely using the intersection
of some polygon to narrow the area to check for as closely as possible.
Since convex hulls are generally very precise or reasonably precise
for a considerable number of applications, and since a convex hull
under scaling/rotation is still a convex hull, it was decided to use convex hulls.
Furthermore, there existed algorithms for finding the intersection
between convex hulls efficiently (namely in linear time).

The earliest prototype was done using Scala and existing libraries.
Scala was used because it can be used with all libraries on the
JVM platform, has high productivity, high safety, high maintainability,
advanced (collection) libraries,
and decent performance compared to C++ and Java.
The library Scalala (Scala Linear Algebra) was used for matrix multiplication,
and the library JTS (JTS Topology Suite) was used for the convex hull intersection.
The Scalala Library was easy to use and performed efficiently,
and was mainly refactored out due to a desire for decreasing dependencies.
The JTS was somewhat easy to use, but didn't perform the convex hull intersection
too efficiently. The reason is likely that JTS is a very general library,
and the convex hull v. convex hull intersection finding was probably supported
by using a more general, but slower algorithm.
Implementing the convex hull intersection algorithm did yield considerable increased performance.

The work with supporting the intersection took considerable time.
However, it was not primarily in the implementation, but in the design phase
that required a lot of time. While the desired algorithm was based upon
an existing algorithm design (see \url{http://www.iro.umontreal.ca/~plante/compGeom/algorithm.html}),
it was an important requirement that the new implementation was geometrically robust.
And that meant handling ALL possible cases correctly.
After a fair bit with experimentation on paper, discovery of cases,
finding general ways to handle issues, etc.,
a possibly correct design was found. As an example of the challenges
of ensuring geometric robustness, in one part of the algorithm about 30 different
cases were identified, which was compressed to about 18 different cases due to
common traits and common handlings.

After some testing and slight debugging (which proved that the overall design
and handling of cases was sane - for instance, one bug was due to the incorrect
identification of a case, and the bugs were fixed by identifying the case correctly),
the prototype was finally done. The testing including textual testing and visual
testing, and some basic performance profiling was also done.

After that, it was decided to port the library to C++.
The reasons include that C++ is close to the metal and
suitable for game library/engine development,
C++ is widely supported by many different vendors and OSs and
C++ is widely used for game library/engine and computational geometry development.
The author had at the beginning not much experience with C++,
nor with porting code from Scala to C++.
Because C++ is a language fraught with dangers and pitfalls,
and the authors experience in C++ was limited,
it was decided to take the 'safe' route and implement
the C++ library in nearly the same way as the Scala library.
While this did give some headaches and trouble,
especially in regards to the type system, the type inference,
the memory handling, the functional programming, templates,
lack of immutable, persistent collections, weird C++-specific behaviours
($vector<bool>$, for instance......for the uninformed,
$vector<bool>$ does not obey the interface for vectors,
due to some weird premature optimization nonsense.
Don't use $vector<bool>$. Use $bitset$, $boost::dynamic\_bitset$ or
$deque<bool>$), it did mean that
once the port was done, the implementation required very little
debugging.

Since the porting was non-idiomatic, it meant that the efficiency
is not nearly as good as it could be, and a simple test
seemed to suggest that the C++ implementation was about 5 times
slower than the Scala implementation for specific cases.
Improving the C++ implementation to become significantly
constant-factor faster should be possible, and it is likely
that making the C++ implementation faster than the Scala
implementation wihtout too much work is likewise possible.

Since the tests was somewhat ad-hoc, it was decided not to
include them in the library.
The library was cleaned up, API-documentation was written,
build systems was used (SBT for Scala, which is very easy,
and CMake for C++), and documentation generation was used
(Scaladoc in SBT for Scala, Doxygen for C++).
These tools generally performed well.

Finally, the manual was written in latex,
images drawn in InkScape,
gitted, and uploaded to github.

