
\section{Intro}

Poxelcoll (\textbf{Po}lygon/pi\textbf{xel}-perfect \textbf{coll}ision detection library)
is a FOSS library designed to support easy, flexible and efficient pixel-perfect collision detection,
including simple transformations such as scaling and rotation.

\subsection{Overview}

This introduction starts with a background that describes the main motivation behind
the library, as well as the main idea of the library. It continues with
simple instructions for basic usage of the library, some general information
and issues regarding collision detection and this library, a part about
geometric and numerical stability, a discussion
of when this library may be useful or not (including potential substitutes).
It ends with a brief discussion of portability, license, status and future
of the library.

\subsection{Background}

2D game development frequently features some sort of physical world where
things can collide. The complexity of modelling and the used solutions
varies greatly, from simple grid-based models, to complex models
with accurate simulation of gravity, friction, momentum, etc.
The part of the modelling that deals with detecting collisions is commonly called
collision detection.
A major challenge for implementing solutions to the more complex models
is \emph{ease of use}, \emph{flexibility} as well as \emph{efficiency}.
The major factor that determines these issues is the basic representations
of collision objects, also known as primitives.
Simpler primitives tend to be more efficient
and easier and not very flexible, while more complex primitives are
much more flexible, but generally not very efficient, easy to use or implement.

For ease of use and efficiency, simple solutions include using
axis-aligned bounding boxes or perfect circles. While these solutions
are very simple, they are not flexible at all, and may limit the gameplay
of the games that use them, solely due to technical limitations.

Another solution which is much more flexible is that of pixel-perfect collision detection.
Pixel-perfect collision detection uses images as a spatial representation of the
collision object. This is conceptually easy, especially if the image used for
the primitive is the same as that which is used for drawing: if a sprite is
seen to collide on screen, it also collides in the game.
Since most 2D games, libraries, tools and engines has some
support for images and sprites, handling images is generally not an issue.
If basic transformations such as scaling and rotation is supported dynamically,
it is also very flexible. Examples of games that uses pixel-perfect collision
detection include the 2D games of the Worm series.

The main issue with pixel-perfect collision detection is that it is difficult to
implement efficiently, especially if basic transformations is supported dynamically.
Simple implementations, without any transformation, goes through each pixel of one
image and checks if it is filled at the same time as the corresponding pixel
in the other image is filled. Even if optimised, this is generally very inefficient.
The problem becomes worse when the images needs to be transformed,
because every time the transformation changes for a given collision object,
the whole image, along with every pixel in it, needs to be changed.
If pixel-perfect collision detection is used or needed for the gameplay,
it can limit the game to small and few collision objects.

A third, general solution is to use complicated or multiple geometric primitives
to represent the collision objects. For instance, convex or concave polygons,
using multiple triangles, boxes, circles, lines, etc. can be used here.
When the desired collision shapes fits well with the geometric primitives,
it can be very efficient.
There are many options, and it can be somewhat flexible, especially if many
primitives is used for each collision object. Transformation is also somewhat
efficient, since geometric primitives are described by relatively few points.

However, it is not always easy to use. If the desired collision shape does not
fit well with geometric primitives, approximating it may take time
and energy, may not be efficient (due to the number of primitives used),
and may not be as precise as desired.

The \textbf{main idea of this library} is to combine the pixel-perfect collision detection
with a general geometric primitive, namely the convex polygon.
The convex hull of the collision object's image is precomputed and stored.
Whenever the collision object must detect collisions with another collision object,
the convex hull is transformed with the basic transformations (rotation, scaling).
Since the convex hull consists of comparatively few points, this will generally
be efficient. The same is done for the other collision object.
With the two transformed convex hulls, a precise intersection between them is found.
This intersection is the only part which needs to be checked for collisions using
the backing images, since the images will definitely not overlap outside
the intersection.
Another advantage is that this enables easy and efficient collision detection
between pixel-perfect collision objects and geometric collision objects.

\subsection{Getting started}

First, pick between the Scala and C++ implementation.
The Scala implementation is (of version 0.1) faster and
has slightly more features, and should be preferred when the JVM
platform is used. The C++ implementation is preferrable when
the platform is not JVM, simply because Scala (as of this writing)
essentially requires the JVM.

Next, read the installation for the given implementation
in the document "INSTALL".

Once the library has been setup and is ready to use,
look at the API documentation for the BinaryImage,
Mask and CollisionInfo classes, in that order.
These are some of the main data types when using the library.
The BinaryImage represents an image, where each pixel
is either on or off. The images that is to be used for
collision detection should be loaded into BinaryImage's.
The mask represent a collision primitive.
It consists of an optional BinaryImage, an origin
around which transformations happen,
a bounding box which covers the whole image,
as well as a convex hull.
For ease of use, factory methods are provided
that automatically generates both the bounding
box and the convex hull given a binary image
and origin point.
The final class is CollisionInfo, which
holds a Mask, some collision information
such as rotation angle and scaling along
x-axis and y-axis, and an id.
While the Mask is meant to be permanent and precomputed,
the CollisionInfo is more temporary.
The id indicates which object the collision object
belongs to.

If images are not always used, the ConvexCCWPolygon must be
understood and used when creating the mask.
Be careful about obeying its contract! Without obeying
its contract, the library gives no guarantees in regards
to correctness or efficiency.

Those classes are the main things you have to worry about.
Once you have those set up, simply use the Pairwise
class to detect collisions. However, it is recommended
that you read or at least skim through the rest of the
manual to understand what is going on, and to know
about the issues in the library.

\subsection{When this library is appropriate}

This library is not suitable for 3D collision detection,
nor is it suitable for platforms where Scala or C++ is not
supported or appropriate.

Furthermore, its license is GPLv3, which may cause problems for some potential users
of this library.

If the issues regarding numerical stability matters, this library is not appropriate at all.
Look into libraries like CGAL.

If pixel-perfect collision detection is not used or needed, there may exist libraries
out there which are more suited to the given context. However, if convex hulls
represented by convex polygons are efficient, flexible and easy enough for the required
purposes, this library may suit the purposes.

If a fully-fledged physics engine is required, this library alone is not enough.
The library has not been built to support or to be incorporated into a physics engine,
and while it may or may not be possible with few modifications, it has not been investigated.
However, if a physics engine with pixel-perfect collision detection is desired or needed,
incorporating this library into an existing physics engine may be the best solution.
For physics engines, box2d is a 2D physics engine written in C++ with a permissive license.

For contexts where the images change, for instance for games where destructable terrain
is featured, the efficiency of this library may or may not be sufficient. The library
requires precomputation of the images for finding the convex hull. This presents two
issues. The first is that the repeated recomputation of the convex hull may be too inefficient.
This depends partly on how frequently images change in the worst case.
The second issue is that images that fitted the convex hull well before the change
does not fit the convex hull well after the change. The effect on performance can be considerable.
Prototyping the library and seeing if its performance is acceptable may be the best option,
but it is in any case recommended that the different aspects that affect performance of the library
is understood clearly before taking any decisions.

For pixel-perfect collision detection systems where basic transformations like rotation and scaling
is not required, it depends on the context whether this library should be used, but for most
contexts this library should perform efficiently. Even if the basic transformations are not used,
this library may be several orders of magnitude more efficient than the naive alternatives
in the best case, and only a low constant-factor slower in the worst case.
See the section regarding efficiency for more information.

Overall, if you need pixel-perfect collision detection, this library (either pure, modified or
combined with another existing library) should be suitable for your purposes.

\subsection{Portability}

NOTE: This pertains to version 0.1.

The portability depends on the implementation used.

For the Scala implementation, there is no dependencies on other libraries,
and the implementation can thus be used on any platform that supports Scala.
The language version used is 2.9.2.
The only external dependencies that may be introduced is that of Swing,
and then only for parts unrelated to the library itself, such as
testing, performance profiling and demonstration.
sbt is used for compiling and generating documentation.

For the C++ implementation, C++11 is used extensively in the implementation.
Therefore, only platforms with compilers that support C++11 can be used.
There are also a few dependencies on Boost, notably dynamic\_bitset
and array.
The make system also happens to be hard-wired to use g++ with compiler
flags for using C++11. Changing the make system for someone experienced
with system administration should in theory be easy, and help would be
appreciated. CMake is used to generate makefiles, and doxygen is used for
generating documentation.

\subsection{License}

The entire library is licensed under GPL v3. For more details, see the document
"LICENSING".

The main reason is that it encourages users to share and improve the library.
By ensuring that everyone who uses the library keeps it open source,
everyone can modify and change it, and improvements from all users can be
redirected to the library, improving the library for all.
Furthermore, it means that the real-world usage of the library can be investigated,
and the library can then be changed and improving based on actual usage and needs.
It also guarantees the freedom of users to use the library as they want to.

Another reason is that a fair amount of work was put into this library,
and it is released in the hope of contributing to the game development
community as well as the FOSS community. Without a proper license,
proprietary users could use this library gratis and without any effort,
thus not contributing anything back.

If you want to use this library in a project, ensure that it is compatible
with the GPL v3.

\subsection{Status and future of the library}

As of version 0.1, 2012-05-08, the \textbf{status} is that the library is somewhat stable and well-documented.
It does not have many features, but does have the most important ones
(efficient pixel-perfect collision detection, efficient basic transformations).
The library has never been used in the real-world, and there are no regression
tests.

The Scala version is more mature than the C++ version.
The reason is that the C++ version is basically a nearly-direct translation
of the Scala source, which means that it is not idiomatic C++, and that
it seems to be a lot slower than the Scala version in basic performance tests.

For the \textbf{future}, three areas can be improved upon.
In the following, they will be described in decreasing priority,
starting with the highest priority.

The first area is stability, correctness and testing of the library.
The reason this is the highest priority is that making the library
correct is non-trivial, and if the library is not correct and robust,
it is not very useful. Issues like geometric stability
and numerical stability makes this area harder to maintain and improve.
More regression tests, demonstrations and performance tests will
make the library more reliable. Real-world usage may also uncover
other issues. In regards to known issues, the best improvement would
be ensuring numerical stability, or at least determine what effects
numerical instability has on the current implementations.

The second area is more features. One possible option is support for
broad phases. This is not a big necessity, since other libraries
already provides this functionality, and pixel-perfect collision
detection does not have different requirements than other collision
detection systems in regards to the broad phase.
Another option is support for only colliding with pixel of a
certain transparency, getting the collision point, etc.
These are smaller features that would take some work,
and if someone needs them, it should be entirely possible
to fork/modify the library to handle them.
Another area is improved precomputation. While the pixel-perfect
collision is generally efficient for images that are accurately approximated
by their convex hull, they are not necessarily efficient for images
that are not approximated well by their conevx hull.
One possible option to handle this is to split the image into smaller
parts, and find the convex hull of these smaller parts.
If the number of parts is small, and these parts are generally
well approximated by their convex hull, this may increase the
efficiency. Providing a general algorithm that automatically
determines if and how to split an image might improve the
efficiency and the ease of use of the library.
Other features would include changing and expanding the
API such that it fits better with real-world usage.

The third area is efficiency, both asymptotic and constant-factor. While the library should be
reasonably efficient, there are room for improvements in many
places. However, flexibility, maintenance, robustness and correctness is more
important, and this area can always be improved upon incrementally
as performance profiling indicates which parts it is worth
making more performant.

