
\section{General topics}

This part introduces general background topics, and how they
pertain to the library.

\subsection{Collision detection}

Collision detection deals with detecting collisions between
collision objects. In many cases, collision detection is a
challenging area, primarily because it is difficult to both
make solutions flexible, easy to use and efficient.
Especially efficiency is very often a large issue.
Since different domains have different requirements in regards
to flexibility, ease of use and efficiency, different solutions
are appropriate in different cases. Nonetheless, there are
some parts of the handling that are general to how the problems
can be approached.

In many domains, the issue of efficiency stems not only from the
complexity and accuracy of the individual collision objects,
but from the possibly large number of different collision objects that may
collide. Since collision objects frequently model spatial objects,
and thus rarely overlap or are all placed on top of each other,
most collision objects are far away from each other.
In some cases, even checking for all objects which objects they
are far away from is too inefficient. For instance, checking
for each of 10.000 objects which of the other objects they are
far away from requires in the brute-force implementation 100.000.000 comparisons.
Depending on the specific context, there exists different
solutions and ways to handle this problem,
having much fewer comparisons than in the naive implementation.
This phase of collision detection is commonly called the broad
phase.

In contrast, the narrow phase of collision detection only deals
with pairs of objects. For complicated objects that are expensive
to detect collisions for, there exists methods such as bounding
volumes to handle it. A bounding volume is one or more over-approximations
of the actual object that are more efficient to check for than the
object itself. The advantage is that if there is no collision,
the checking of the bounding volumes may determine that there is
no collision, and thus the more expensive collision detection is avoided.
Variations on this includes bounding volume hierarchies.
In this library, the over-approximating bounding volumes is primarily
the axis-aligned bounding box and the convex polygon for the convex hull.
The axis-aligned bounding box is less precise but faster, while the
reverse is true for the convex hull.

This library deals exclusively with the narrow phase as of version 0.1.
If the broad phase of the collision detection is not efficient enough,
this library can be combined with other libraries or solutions
that handles it better.

\subsection{Geometric and numerical stability}

Geometric stability and numerical stability are two important properties
in computational geometry. Since this library contains elements of
computational geometry, it is important to discuss these properties
in regards to the library.

Geometric stability is about considering and handling all geometric cases correctly
in a computational geometry algorithm, no matter what kind of input is given.
As an example, an algorithm that finds the intersection between two convex polygons
that finds the correct answer most of the time, for instance when the intersection
is a polygon itself, but fails in some way (crash, wrong answer, infinite loop, etc.)
for the cases where the intersection is a line, a point or is empty is not geometrically
stable. Conversely, an algorithm that always finds the correct answer no matter what the
intersection is, is geometrically stable.
All algorithms in the library are designed to be geometrically stable, and seems to
be geometrically stable in tests.

Geometrical algorithms frequently assumes that calculation with numbers has infinite
precision. This is not true in general. When an algorithm that would be correct
if the precision was indeed infinite, but is wrong when it isn't, it is because the
algorithm is not numerically stable. This is a big problem in computational geometry
because even small errors can have large consequences. While in other fields finite
precision for numbers mostly means decreased precision, it can in computational
geometry lead to errors, wrong answers, crashes, infinite loops, etc.
An example is testing the cross product of two vectors.
If a rounding error gives a wrong result (ie. negative result instead of positive),
a different decision branch of the algorithm may be taken.
The consequences and effects of numerical stability are fully unknown as of version 0.1,
and even though they have not caused problems in tests so far, it does not mean
that they cannot produce undefined behaviour given certain input.
For this reason, this library should not be used for scientific or mission-critical
purposes, nor for any purposes where the potential consequences of numerical stability
are not acceptable. This issue may or may not be improved upon in the future.

